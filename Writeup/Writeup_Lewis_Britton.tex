\documentclass[11pt, english]{article}
	\usepackage{geometry}
 		\geometry{
 			a4paper,total={210mm,297mm},
 			tmargin=40.8mm,
			bmargin=40.8mm,
			lmargin=32.6mm,
			rmargin=32.6mm,
 		}

	\usepackage{fancyhdr}
	\usepackage{lipsum}
		\pagestyle{fancy}
		\fancyhf{} 
		\fancyhead[L]{\leftmark}
		\fancyhead[R]{\thepage}
		\fancyfoot[C]{\thepage}
		\renewcommand{\headrulewidth}{0.5pt}

	\usepackage{tocloft}
		
		\renewcommand{\cfttoctitlefont}{\fontsize{18}{16}\scshape}
		\renewcommand{\cftlottitlefont}{\fontsize{18}{16}\scshape}
		\renewcommand{\cftloftitlefont}{\fontsize{18}{16}\scshape}

		\renewcommand{\cftsecfont}{\scshape}
		\renewcommand{\cftsubsecfont}{\scshape}
		\renewcommand{\cftsubsubsecfont}{\scshape}
		\renewcommand{\cftparafont}{\scshape}

	\usepackage{abstract}
		\renewcommand{\abstractnamefont}{\fontsize{11}{0}\scshape}

	\renewcommand{\thesection}{\arabic{section}}
	\renewcommand{\thesubsection}{\thesection.\arabic{subsection}}
	\renewcommand{\thesubsubsection}{\thesubsection.\arabic{subsubsection}}
	\renewcommand{\theparagraph}{\thesubsubsection.\arabic{paragraph}}

	\usepackage{titlesec}

		\titleformat{\section}
			{\fontsize{18}{16}\scshape}{\thesection}{0.5em}{}

		\titleformat{\subsection}
			{\fontsize{14}{16}\scshape}{\thesubsection}{1em}{}

		\titleformat{\subsubsection}
			{\fontsize{11}{16}\scshape}{\thesubsubsection}{1em}{}

		\titleformat{\paragraph}
			{\fontsize{11}{16}\scshape}{\theparagraph}{1em}{}

	\usepackage{hyperref} 
		\hypersetup{          
        		colorlinks=true,        
        		linkcolor=black,  
        		filecolor=magenta,
        		urlcolor=cyan,
        		}

	\usepackage[labelfont=sc,textfont=sc,font=small,skip=8pt]{caption}

	\usepackage{float}

		\renewcommand{\thetable}
			{\thesection.\arabic{table}}

		\renewcommand{\thefigure}
			{\arabic{figure}}

	\setlength{\parindent}{0pt}

	\renewcommand{\baselinestretch}{1.25}
	\usepackage{setspace}

	\newcommand{\HRule}[1]{\rule{\linewidth}{#1}}
		\setcounter{tocdepth}{5}
		\setcounter{secnumdepth}{5}

	\usepackage{longtable}
	\usepackage{multicol}
	\usepackage{multirow}

	\usepackage{amsmath}
	\usepackage{amssymb}

	\usepackage{graphicx}
	\graphicspath{{./Figures/}}

	\usepackage{tipa}

	\usepackage{babel}

\begin{document}

% Title Page

\pagenumbering{gobble}

	\title{
                \HRule{0.5pt}\\ [0.3cm]
                \huge\textsc{CS958 Project}\\
                \Large\textsc{Coursework Assignment}\\ [0.25cm]
                \HRule{0.5pt}
                }
	\author{\textsc{Lewis W. Britton}\\
                \textsc{202194412}\\
                \textsc{University of Strathclyde}\\
		\textit{Glasgow City, Scotland}
                }
	\date{}
	\maketitle

        \begin{center}
                \textsc{Burning Roots: }
        \end{center}

        \vspace{\fill}

	\begin{center}
		\textsc{Written \& Directed By}\\ \textit{John Hughes}\\
		\textsc{Executive Producer}\\ \textit{Michael Mann}\\
		\textsc{Created By}\\ \textit{Anthony Yerkovich}\\
		\textsc{Music Composed \& Performed By}\\ \textit{Jan Hammer}
	\end{center}

	\begin{center}
		\fbox{\textsc{... Words}}
	\end{center}

	\begin{center}
        	\textsc{Dissertation Submitted in Partial Fulfilment of the Requirements for the Degree of Master of Science Software Development at the University of Strathclyde}
	\end{center}

	\begin{center}
		\textsc{Academic Year 2020/2021}
	\end{center}

\newpage

\pagenumbering{roman}

	\begin{abstract}
	\end{abstract}

	\textsc{Index terms:}

\newpage
% Declaration

	\section*{Declaration \& Information}

	This dissertation is submitted in partial fulfilment of the requirements for the degree of Master of Science Software Development at the University of Strathclyde. It accords with the University’s regulations for the programme as detailed in the University Calendar.

	\begin{center}
		\small
	\begin{tabular}{p{5.45cm}|p{5.45cm}}
		\textsc{Mail To:} wi.lbritton@yahoo.com & \textsc{Telephone:} 07415 212 ***\\
		\textsc{Website:} \href{http://lewisbritton.com}{lewisbritton.com} & \textsc{GitHub:} \href{https://github.com/FedeRog1977}{FedeRog1977}\\
	\end{tabular}
	\end{center}

	This document’s presentation reflects the use of {\LaTeX} typesetting (Figure B7), using Computer Modern Unicode (Figure B8) (I haven't reached my GNU Troff phase yet). This escapes the inane formatting requirements of my institution. References are presented using \textsc{Bib}{\TeX}, favouring \textsl{oblique} over \textit{italic}, in-line with Donald E. Knuth’s preference (Knuth, 2020). The process is executed in command line using Vim, which is a very powerful editor that has many commands, too many to explain in a tutor such as this. For maximum optical pleasure, the use of M$\mu$PDF is vigorously advised with [-I]. Navigate this document using h($\leftarrow$), j($\downarrow$), k($\uparrow$), l($\rightarrow$), ensuring that the Caps-Lock, Super-Key `mod', or any other command key is not depressed. Note that the Oxford Serial Comma is favoured throughout this text. This study's sentence structure focuses on pragmatics and syntax, disregarding bloated filler content. Arguments are coherent, logical, definitive and straight-to-the-point. Nugatory theory is ignored. If you are curious about any of the mathematical, operational, logical, etc., symbols or notation used in this report, a comprehensive {\LaTeX}-syntax-based symbolist will be available from my \href{http://lewisbritton.com/Library.html}{website library} from approximately summer 2021.\\
 
	The word count of this piece reflects relevant content from titles, heading classes 1, 2 and 3, paragraphs, footnotes, tables (excluding [results] tables 4.7, 5.10 $\rightarrow$ 5.20), table titles, figures, and figure titles in \textit{Chapters 1 $\rightarrow$ 6}. Word count excludes any pre/succeeding content from \textit{Abstract}, \textit{Declaration \& Information}, \textit{Acknowledgements}, \textit{Table of Contents}, \textit{Appendices}, and \textit{Bibliography}.\\

	I declare that this document embodies the results of my own work and that it has been composed by myself. Following normal academic conventions, I have made due acknowledgement of the work of others.\\

	Signed:\\ 

	Date:

\newpage
% Acknowledgements

	\section*{Acknowledgements}

I would like to thank my dissertation supervisor, Dr Devraj Basu, for his approach with regards belief that students must be self-disciplined, organised, structured and punctual to their own degree. This closely relates to my own personally practiced work ethic and philosophy of the ASAP standard.\\

I would like to give credit for the computational aspect of this study to one of my biggest inspirations, John ``The Tzar'' Kelly. He inspired my love for everything barebones computational, from simple arrays (of hope), through Hyperthreading-enabled, all the way to x86 Assembly. I would also like to accredit Luke Smith for the foundation of my knowledge of Bram Moolenaar’s Vim and {\LaTeX}. This study’s presentation would not have optimal without Smith (2015).\\

This piece would not have been as efficient without the aid of the only acceptable Linux distribution, `distro’ if you will, Arch Linux. I would like to thank Judd Vinet for his eye-opening and life-altering contribution to the development and computer-system enthusiast community. ‘The Arch Principal’ is certainly out in high force. Finally, for making use of this software mechanically efficient, I would like to thank IBM for the creation of the ThinkPad T23, X30, T42, R50e, T60, X60, X200, X220 and T420 neoVimPads, the UltraDock, and the 1987 Model M \textit{Catastrophically Buckling Compression Column Switch and Actuator} typehorse (US369 9296A, 1972). For your convenience, one of my \href{http://lewisbritton.com/Blog/ThinkFlow.html}{blog posts} can satisfy your interest in this. 

\newpage

	\renewcommand{\contentsname}{Table of Contents}

	\tableofcontents

\newpage

	\listoftables

\newpage

	\listoffigures

\newpage

\pagenumbering{arabic}

\section{Introduction}\label{ch1}

	\subsection{Purpose \& Industry}

	The system developed throughout this project is a functional web application based on providing user-location-based and external GPS data. User-based GPS services are based on data relevant to the user's machine and external services are based on context-relevant data. The system, which so forth may be referred to as `the system', `the application', `the site', provides detailed information, guidance and recommendations relevant to sport and leisure, particularly hiking and its associated practices, in Scotland's fine rural outdoors and [\textit{these mist covered mountains, which are home now for me}]. Therefore, upon a hypothetical release of a full version of this system, it would be a direct competitor of services such as Walkhighlands, Strava, Garmin Connect, AllTrials, etc. Due to the autistic and comprehensive nature of its development, it would be in the market not in the competition-driven business, but in the [\href{https://youtu.be/c18_Thy6kJo?t=204}{empire business}].

	\subsection{System Structure}

	The site upon which the system is spread is static, not dynamic, meaning any possible requests made buy the user are based on existing data. The site does not reference an external database for any purpose. Thus, no PHP or SQL-relevant content. The system is segmented into four, approximately equivalent, parts with the metric(s) determining their weight being algorithmic volume, number of services, etc.\\

	The first, `home'/`drafting room', page allows the user to view a comprehensive overview of the site's offerings. It allows the user to quick-view activities in the `overview' section; analyse their projected personal ability and their projected gear performance (upon their input(s)) in the `conditioning' section under `ability' and `equipment cache'; and, view a coordinate-based weather briefing in the `weather' section. All of these sections include `key's and `suggested reading's. Yes, this site is that obnoxious.\\

	The second, `conquest map', page allows the user to view and interact with various GPS and mapping features including their location and location seeking. It also includes functionality which allows pinpointing of particular features upon interaction, such as Munros, Munro Tops, Corbetts, Corbett Tops, etc. Further it includes aspects which allow the user to seek, be recommended, select and print GPX routes on the map.\\

	The third segment is the `ranger calculator' which is used purely for analytical purposes and allows the user to input data relevant to their ability, equipment, routes, etc. and will deliver output in the form of statistics tables and charts based on various computations. This element does not implement any GPS functionality in-line with the system's primary focus however, is extremely relevant.\\

	The final segment is the `general search' function which allows the user to input or select search criteria which returns a comprehensive overview of all statistics relevant to the match(es). Unlike the other sections, this is more subjective and informative, as opposed to being logical / statistics-oriented. That is, is exists more so for the users understanding of what they're doing and how to actually interpret some of the statistics they're being delivered in other elements. It is open to their use and interpretation.\\

	The service is called `\textit{Burning Roots}'. No, the rhetorical use of satiric misspelling is not unintentional malapropism; it is in fact deliberate. In harmony with the feeling you'd experience when [racing south-west to Lone Stallion Ranch], the term references that certain `\textit{burning} desire' for freedom in the sweet country air, the one you only experience when digging to your deepest `\textit{roots}' to achieve a new personal record or firmly assert your dominance over your inferiors. This service uses and computes data to encourage a user to take to the trails with motivation to be the fastest, most efficient, most prepared and most endured athlete on their \textit{routes}.

	\subsection{Users \& Platforms}

	All users of this system will be sport/leisure oriented and as this is focused on a specific group of enthusiasts who have a firm set of beliefs and a strong pre-developed relationship with their sport (lifestyle), it will likely only receive traffic from athletes who are already hiking-inclined. It may encourage new hikers due to the comprehensive nature and customization opportunity of the learning and planning material however, it is unlikely. The most efficient empires dominate only one type of market. This market does however expand to: walkers, hillwalking enthusiasts, scrambling-inclined 4x4s, [T6 vanlife] climbers, and Scottish mountaineers / ice climbers.\\

	Upon original briefing, this system was planned to include a road cycling section which would essentially mirror the hiking part with cycling-specific data. However, upon reflection this element is irrelevant for two reasons. The first being all functional areas are covered by the computational processes involved with hiking data. And second, road cycling has little association with hiking and therefore the adjacent sport may be seen as irrelevant by a specific user. If it were to be included, it would only be logical to implement a wider array of sports for example, excluding road cycling and including mountain biking and (fell) running, which are actually relevant to hiking. Or an even wider array if road cycling were to blend in seamlessly. This is unnecessarily [time-consooming] and only duplicates processes and would not deliver additional benefit, only diminishing returns, upon the project marking process. I'm not [the Zuck'], not only do I not have the time or resources for this, I do not have the relevant background knowledge.\\

	As far as platforms go, the user arrives at this site through a web browser. This system is deployed as a website and is therefore extremely versatile and usable on any device. Browser caching of script elements allows a user to view and interact with the relevant data offline, provided they receive GPS signal. For example, they can still view their location and route on the `conquest map'. Of course, [this means that] the site is constructed using HTML, CSS and JavaScript. Elements of JavaScript allow this site to dynamically scale to various device sizes tailor relevant content to these devices.

	\subsection{Development Process}

	As this is a [solo project], it's one man, his [ThinkPad X220], [Artix Linux] and his [neoVim] setup. There is little requirement for extensive use of any formal [inane] project management methods such as team-based allocations or associated time-based or progress management coordination frameworks. Therefore, any adherence to processes aimed at mapping management of this project are / have been more logic-oriented and variable, allowing creative freedom. I work on an ASAP basis so one creative day of thinking may be followed by a [5am -- 11pm] of implementation, which may then be followed by 2 days of idling. Any formal micro-level plan would be redundant.\\

	Succeeding acquisition of user requirements, the most important part of gaining an understanding of how this system would look and operate is determining how the user interacts with the various aspects of their sport. That is, what data/inputs must the user provide, how will this be used, and what will it be used for to satisfy the requirements. Therefore, the mapping of the functional structure of the user interface (UI) leads this in the sense that it demonstrates the logic and process of interaction relevant to this data. Thus, this creates a valid starting point. And so forth, the development process of this project reflects what follows:

	\begin{center}
		Requirement analysis\\
		$\rightarrow$ Structural design\\
		$\rightarrow$ Usable data construction\\
		$\rightarrow$ Graphical user interface functionality\\
		$\rightarrow$ Usable data implementation\\
		$\rightarrow$ Graphical user interface graphic design\\
		$\rightarrow$ Testing
	\end{center}

	Following the structural mapping, there is little sense in proceeding without any data to work with as incremental testing of site functionality would be challenging to impossible. So, it's at this point which the acquisition of relevant data takes place. In this case, this data accounts for non-user-centered data such as GPS coordinates, map regions, landmasses and their attributes, etc., which are essential for the majority of computations. Therefore, not only is [JavaBloat] implemented to manage site dynamics, it also computes based on data from these discussed files, in JSON format. It is only after this when the graphic design of the site can be allocated more focus, however of course much of it comes instinctively along the way also. After this, and frequent incremental developer tests, the system is ready for more expansive developer and user testing.\\

	For the natural ease in workflow, for the developer's mental state, and for the minimization of [nugatory] methodologies and [bloated] task flow [cargo donkeys] such as IDEs and [froymeworks], all files (including `code', data files, notes and write-up) are [composed and performed by Lewis Britton] in [Bram Moolenaar's Neo Vi Improved], in the command line of a pragmatic dwm setup on Artix. HTML, CSS and JavaScript is written from scratch in plain text format, therefore using no environment prompts or assistance, in order to keep the process practical. All write-up documentation is transcribed using [Donald `Don' E. Knuth's \TeX]. Or as some [neomoderinists] like to use, \LaTeX. Due to time constraint, there is unfortunately no mastering the fine art of [GNU Troff], so transcriptions may not appear [optically optimal] without the famous Groff-PostScript [multi-kill].

	\subsection{Disposition}

	So forth, the following elements of this project are responsible for...

	\textsc{2 Research \& Evidential Background}\\

	Explores the areas of research and data gathering including hiking routes, hiking equipment, personal fitness and ability, geography and geology. Furthermore, presenting and examining results and conclusions to evaluations of currently existing competitors' services. This section acts as a literature review would in a paper based on, say, an empirical piece investigation; providing the foundational material upon which development aims to further succeed and `develop'.\\

	\textsc{3 Data Processing \& Methodology}\\
	
	Mapping and justifying the data selected for use in the system. This is broken into three segments, first being data acquisition which explains how and why data is sources from third parties and inputted from users. The second section explains how this data is manipulated and the third; statistical and informative output.\\

	\textsc{4 System Requirements}\\
	
	Describing the scale and scope of the users and their requirements for this system, and mapping how these are prioritized at the beginning of and throughout the project. This is in the context of functional and non-functional requirements.\\

	\textsc{5 System Design}\\
	
	Displaying the structure of the system architecture and how the logic aligns with the requirements of the system. Also, describing the various aspects of the user interface's functional and graphical communication and design process. It's apparent at this point how the data structure is made relevant to the design of the system using the requirements.\\

	\textsc{6 System Construction \& Implementation}\\
	
	Providing a closer look at and justification of the development environment, languages and protocols selected for the creation of this system and exploring the various APIs and JavaScript libraries and other supporting tools used to enhance the system and allow it to function in harmony. Also, Providing an overview of how these elements were implemented from a project management point of view.\\

	\textsc{7 Evaluation}\\
	
	An evaluation of requirements gathering and the feasibility and tangibility of their implementation, an review of self-testing methods and additional tests, demonstrations of prototyping and various other aspects of developer' and user-centered testing.\\

	\textsc{8 Development Conclusions}\\

	Summaries of development conclusions which are presented pragmatically as objective, critical notes and possible segues.

\newpage

\section{Research \& Evidential Background}\label{ch2}

	\subsection{Areas of Exploration}

	This system is designed to combine and present aspects of the different types hiking routes and their attributes relative to their conditions; the recommended and available equipment for users to investigate and explore expansive opportunities within; the personal fitness and ability level of users and therefore, their ability to interact with different routes and opportunities; and, the geography and geology of various aspects of hiking routes which contributes to various other factors within user ability how users may interact with the routes themselves. 

		\subsubsection{Hiking Routes}

		The hiking routes are the foundation of this system. They provide the purpose and reason to the GPS aspect of the system. There are various demonstrations of how hiking routes are implemented in different ways across slightly different platforms. Walkhighlands for example, presents very static use of these; displaying a page per routes listing manually expressed data and literature. Each route cannot be interacted with and had no dynamic attributes. They are simply listed for user interpretation. In this system, GPS routes are made relevant to the particular user interacting with them. Routes are not only selected through subjective choice, they are dynamically relevant to both user conscious and subconscious attributes.

		\subsubsection{Hiking Equipment}

		Hiking equipment is arguably half the battle when it comes to most effectively tackling projects. Although I've had my fair share of 15 mile proj's with approach shoes and one litre of water, alongside [Griff] in shredded boots and MTB tee-shirt and shorts; it's still pretty important. To the [Maddie Owens] of the industry. Regardless, many routes require particular components and combinations of equipment, including the appropriate knowledge of such. This means it is completely necessary that, especially under-experienced hikers, are as aware of the precautions and hazards present on selected and suggested routes. Including a metric which accounts for the user's equipment, alongside relevant literature, ensures that this site takes the implements the correct protocol to see that the user does not make any unrealistic inference regarding routes. Once again, in other services such as Strava and Garmin Connect, there are not metrics which account for these attributes. Within Walkhighlands, there is plenty of literature available however, this information is not quantified and translated into data input so therefore leaves routes static, relative to equipment.

		\subsubsection{Personal Fitness \& Ability}

		Fitness is undoubtedly the single most important factor in any sport. Skill, knowledge, technique, understanding of kinematics and dynamics of the human body, and things alike all contribute to the degree to which you excel at a sport. However, without raw fitness you might as well sit on the bench. Keeping the heart rate regulated, understanding which parts of your body to engage and not to engage, correctly distributing force and converting torque are all closely related to personal fitness and therefore must be quantified in such a way which reflects a user's expected effort and ability to complete a route. This effects results such as elapsed time, breaks required, fatigue and estimated recovery time, etc. Again, route planners such as Walkhighlands do not offer any form of input using these metrics. Strava and Garmin Connect do however make estimates following completion of activities however, do not allow these statistics to be re-used and inputted as variables determining results of estimates of future activities.

		\subsubsection{Geography}

		The use of geography within this system is fairly static and informational. That is, it is not quantified and it's attributes cannot be used as inputs which determine future estimates and results. As physical geography is out of the control of user's however, there wouldn't be much use in quantifying it. It is however useful if users have an understanding of what geographical features are and how they can have an impact on their routes. Of course this particular feature is irrelevant to much of Strava and Garmin Connect's functionality and is therefore not included in any form. Walkhighlands does include excellent educational sources however there is not much continuity and consistency to their presence. Therefore, relevance is often unaligned.

	\subsection{Material Investigation \& Heuristic Evaluations}

	As discussed, improving upon various features and aspects of Walkhighlands, Strava and Garmin Connect is a relevant step in developing requirements and informal desires from this system. It goes without saying that as an inexperienced sole developer, these improvements are not based on functionality and code efficiency etc. This would be intangible. Improvements are primarily focused on making particular features more relevant, accessible and usable. The three services under examination are significantly more advanced and expansive than this system is at the end of development. Therefore, features which these services include but this system does not will not be examined. So forth, the services will be examined purely under the scope of this system. That is, basic services (overview, ability, equipment) integration; map services (OS map, GPX file and map feature) integration; and, statistical processing (route and mountain information, and personal data processing).\\

	As an alternative to formal empirical methodologies, Nielsen (1994) proposes a critique-based method based on a heuristic evaluation which involves analysis based on areas of expertise. A heuristic evaluation of one's own system is also argued to be a useful method of allocating time to minor issues before final user testing. Nielsen also claims that the optimal number of `experts' assigned to an evaluation is three-to-five in order to find the `optimal' number of issues relative to the cost-benefit analysis. In this case however, one examiner is used for obvious reasons. Each issue is individually listed and valued against the set of ten heuristic factors and assigned a severity rating, as seen in Table 2.1.\\

	The analyses conducted subsequently are not exhaustive however, are relevant to the context and features in this system. There is no heuristic evaluation for Garmin Connect as any differing functionality from Strava is more advanced than that which this system accounts for and therefore, does not need to be evaluated.

	\begin{table}[h]
                \scriptsize
                \renewcommand{\arraystretch}{1.25}
        \begin{center}
        \begin{tabular}{c}
                \hline
                Heuristics\\
                \hline
                \multicolumn{1}{l}{$\mathrm{H_{1}}$: Visibility of System Status}\\
                \multicolumn{1}{l}{$\mathrm{H_{2}}$: System-Real-World Match}\\
                \multicolumn{1}{l}{$\mathrm{H_{3}}$: User Control \& Freedom}\\
                \multicolumn{1}{l}{$\mathrm{H_{4}}$: Consistency \& Standards}\\
                \multicolumn{1}{l}{$\mathrm{H_{5}}$: Error Prevention}\\
                \multicolumn{1}{l}{$\mathrm{H_{6}}$: Recognition Rather than Recall}\\
                \multicolumn{1}{l}{$\mathrm{H_{7}}$: Flexibility \& Efficiency of Use}\\
                \multicolumn{1}{l}{$\mathrm{H_{8}}$: Aesthetic \& Minimalist Design}\\
                \multicolumn{1}{l}{$\mathrm{H_{9}}$: User Recognition, Diagnostic \& Recovery from Error}\\
                \multicolumn{1}{l}{$\mathrm{H_{10}}$: Help \& Documentation}\\
                \hline
                Severity Ratings\\
                \hline
                \multicolumn{1}{l}{$\mathrm{S_{0}}$: Don't think it is a usability problem}\\
                \multicolumn{1}{l}{$\mathrm{S_{1}}$: Cosmetic issue; repair in additional time}\\
                \multicolumn{1}{l}{$\mathrm{S_{2}}$: Minor usability problem; allocate low priority to repair}\\
                \multicolumn{1}{l}{$\mathrm{S_{3}}$: Major usability problem; allocate high priority to repair}\\
                \multicolumn{1}{l}{$\mathrm{S_{4}}$: Critical error; repair immediately}\\
                \hline
        \end{tabular}
                \caption{Heuristics \& Severity Rating}
        \end{center}
        \end{table}

		\subsubsection{Walkhighlands}

	The heuristic evaluation for Walkhighlands is listed under Figure A2 in Appendix 1.

		\subsubsection{Strava}

	The heuristic evaluation for Strava is listed under Figure A3 in Appendix 1.

\newpage

\section{Data Processing \& Methodology}\label{ch3}

	\subsection{Data Acquisition}

	A program has no applicable context without the appropriate data. In this case, data is required to even present the base functionality and content related to many features. That speaks loudly for the fact that this system is primarily made of JavaScript and provides users with no utility as a pure site. At least it's static however, if that's any compensation. Seen in Figure A1 is a brief overview of all the data sources, purposes and uses throughout every component of this system, which are discussed under the seven subsequent sub-sections of this section.

		\subsubsection{Location}

	A digital mapping system for mountain navigation will never function as required if no location services are in place. And, at the heart of location tracking with map APIs and other services alike, is Geolocation. Once a call to \verb|navigator.geolocation| has been made which interfaces with the browser being used, the user is prompted to grant access to device location services. Once this is confirmed, GPS data is easily accessible. The two important values captured using Geolocation are of course latitude and longitude (\texttt{lat}, \texttt{lon}), which are used in two important ways. The first being \verb|Geolocation.getCurrentPosition()| which is useful for using latitude and longitude in calculations such as measures of distance, historic placement tracking, etc. The second is \verb|Geolocation.watchPosition()| which is a method of tracking location constantly upon update, clearly more useful for icon display. This is the simplest form of data collection in the this system as it only requires an API call and perhaps a few more lines to manipulate a result.

		\subsubsection{Regional}

	The regional data refers to GPS coordinate records of [1] all counties of Scotland, [2] all regions of Scotland, [3] all sub-regions of Scotland, and [4] all sub-sub-regions of Scotland. For example, \textit{Moray} is a County; \textit{Glenfinnan} is a sub-sub-region, in the sub-region \textit{Fort William, Lochaber and Lorn}, in the region \textit{Highlands}. All of this regional data was acquired by inspecting the `Search all walks' section of Walkhighlands (Walkhighlands, 2022), although did require some hierarchical and grouping amendments, especially throughout the [Western Ocean] and Western Isles. It is stored in a JSON file created locally by myself and hosted on my GitHub.

		\subsubsection{Landmass}

	Landmass data is of a similar format in that is stores records of GPS coordinate data upon the landmasses of Scotland. However, it extends to be far more complex. It includes data on landmasses themselves, which are anything resembling some significant land-form of Scotland. In this context, a landmass may either be a \textit{Mountain}, \textit{Mountain Range}, or \textit{Stand Alone}. A mountain is a single prominent feature which contains one or more hills without a contour drop of under 2000ft between summits; a mountain range is a group of prominent features which contain hills without a contour drop of under 1200ft between summits; and, a stand alone is a single prominent feature which contains one hill. `Hill' refers to \textit{Munros} and \textit{Corbetts}, not \textit{Munro/Corbett Tops}. Therefore, tops are not limited on stand alones but also do not determine a summit. For example, ``a stand alone landmass may have one Munro and two Munro Tops''. Note that \textit{Munro Tops} and \textit{Corbett Tops} sit at the height of Munros/Corbetts respectively however do not meet the elevation and distance criteria to be classed as their own hills. They must be present at maximum stationary points, not increasing points such as those at which cairns are often found. All attributes of landmasses and hills etc., and their hierarchy can be found in the breakdown of the JSON file in Figure A1.\\

	Data was gathered, wait for it..., primarily manually by myself by staring at OS Explorer 376\footnote{Oban \& North Lorn}, Explorer 377\footnote{Loch Etive \& Glen Orchy} and Explorer 384\footnote{Glen Coe \& Glen Etive}. This provided access to information not yet recorded, such as Corries, Lochains and boulder fields of a landmass, for example. To ensure inclusion of all Munros and Corbetts, reference to Walkhighlands was made (Walkhighlands, 2022); and, to ensure inclusion of all Munro and Corbett Tops, Harold Street (2022) and Peakbagger.com (2022) were also investigated as they include more comprehensive community-driven data. Once again, data gathered is stored in a JSON file hosted on my GitHub.\\

	Under the time constraint of the project, some omissions, amendments and notes were made as follow:

	\begin{itemize}
	\setlength\itemsep{0cm}
		\item The sample size is limited to the South-West Highlands, which refers to the region spanning Glen Etive, Glen Orchy and Glen Coe, a.k.a. the land between the A85 to Oban and the A82 through Glen Orchy, through Glen Coe and Ballachullish (side note: don't confuse `Glen Coe' with `Glencoe' in the context of this data! Critical Errors will occur).
		\item Sub2000s, Donalds, Grahams and Grahams tops do not appear in the data as this would have taken an inappropriate amount of time to collect and the six-thousand-line JSON was enough already.
		\item Some details for Corbett Tops and Munro Tops, are not provided as even the most detailed aforementioned sites were lacking some data.
		\item Munros, Munro Tops, Corbetts and Corbett Tops are ordered primarily by category and secondarily from highest-to-lowest within their category.
		\item There is no pre-determined `order' to Corrie hierarchy on landmasses so they are simply ordered by myself from south-to-north.
		\item Ideally, Corries, Gullies, Lochains and Waterfalls would be named and assigned GPS coordinates in the JSON file. However, OS maps were too inconsistent with naming the features so only the 100\% consistent Corrie category is given this luxury. Others are simply assigned boolean values.
		\item HTML symbols such as \`{o} (\verb|&ograve;|) and \^{e} (\verb|&ecric;|), which appear frequently in Gaelic spellings, are unfortunately not favoured in the JSON or any other iteration of the relevant words, as much time is saved developing functions and methods to search; loop through and match the data.
	\end{itemize}

		\subsubsection{Route}

		\subsubsection{Ability}

		\subsubsection{Equipment}

		\subsubsection{Weather}

	\subsection{Data Manipulation \& Output}

		\subsubsection{`Overview'}	

		\subsubsection{`Conditioning'}
	
		\subsubsection{`Weather'}

		\subsubsection{`Conquest' Map}

		\subsubsection{`Ranger' Calculator}

		Risk indicators: lack of visibility (based on weather), misdirection (based on weather and terrain), falling off cliffs and ridges (based on weather and terrain), rock fall (based on terrain), rock kick-back (based on terrain), dehydration (based on weather and duration etc.), sunburn (based on weather and duration etc.), confrontation with livestock etc. (based on terrain and terrain type), plants and allergens like moss and ferns etc. (based on terrain and terrain type)\\

		Winter risk indicators: lack of visibility - white-out (based on weather and terrain), cornices (based on weather and terrain), avalanche (based on weather), ice fall (based on weather), hypothermia (based on weather and ability)\\

		Ranger graphs: elevation profile, speed input, power input, heart rate input; Select: Constant Speed (Max); Output: Required Time and Power; Select: Constant Speed (Average); Output: Required Time and Power; Select: Constant Power Output; Output: Required Speed and Time; Input: Target Speed; Output: Required Time and Power; Input: Target Power Output; Output: Required Time and Power; Input: Target Energy Output; Output: Required Time and Power

	\subsection{Information Acquisition \& Output}

		\subsubsection{Purpose}

		\subsubsection{Keys \& Suggested Reading}

\newpage

\section{System Requirements}\label{ch4}

	System requirements gathering is essentially the stage at which the concept of a system's functionality is aligned with real-world user desires (requirements). This stage provides a context for creativity and a blueprint upon which this can be mapped. Although the founder of a system or concept may have a clear vision of the intended outcome of their development, understanding what final users need and want allows the creator to constantly tailor development. This may extend to how they and/or the software could/should gather information, store data, transact data, and output data/information.

	\subsection{Scale \& Scope}

	I repeat, this is a [solo project] so there are some restrictions regarding the overall scalability of the project as a whole. If this system were to be designed by a team of professionals, it would be very large-scale and offer much expansible functionality. However, due to the number of personnel assigned to the task (a.k.a. me), the time constraint, and budget constraint, the system finds itself with two major general down-scales: [1] there is no user-data back-end, meaning user accounts are unavailable on this system, which is acceptable as the static functionality of the site is most relevant; and, [2] the sample region for data collection is significantly smaller than that offered by other services, which is also acceptable as adding a wider scope of data (to the master JSON in this case) would only consist of repeating the same patterns perpetually. As this system does not aggregate this data for any form of cross-sectional statistical analysis, larger sample sizes become irrelevant after a certain point. Overall, this implies these factors are not directly related to any implemented requirements. Many of the specifics of these restrictions, and others, are discussed latterly in the \textit{Evaluation} section.\\

	In an ideal world, users would be able to expand their scope of interactions within the system (i.e. different sports with a wider and varying array of attributes), and the individual scale of these. [This means that] as sports differ, attributes and statistics differ, and information and guidance differs; offering a better-rounded service. Ideally, users should have the opportunity to fully customize their experience however, to do this on such a scale feasible with the development team available (myself) would be a significant over-effort for an under-achievement. This is why user accounts have been disregarded. This decision helps keep the system and implementation of requirements more manageable and makes it easier to achieve a polished product within the time constraint. 

	\subsection{Gathering \& Prioritization Methodology}

	In this scenario, there are three methods of requirements gathering and inference. The first is the purely user-centered method. As I am surrounded by people who share an interest in the form of this system, including professional developers, student developers and various other [NPCs] of the sort, they act as an accurate representation of market users as they share the same attributes. The second mode is alike however, is argued to be subject to various aspects of contextual bias. This consists of creative direction explicitly from the creator, me. In this context, these inputs will generally align with those of the formerly discussed however, due to the bias, is not considered a viable user-centered method unless used in conjunction with others affirmative of said criteria. The final method involves basic inference from the \textit{Research \& Evidential Background} section. That is, much of the material investigation of Walkhighlands, Strava and Garmin Connect in this section highlights areas for linear development. These refer to aspects which generally do not require user-centered input and must be developed purely functionally.\\

	Many aspects of this system are purely functional and exist to serve an objective purpose. For example, relaying GPX and JSON data related to components such as route and weather information. [This means that] the primary functionality of the system is majorly accounted for in the latter of the three requirements gathering methods, in that the goal of the system is to create wider-scoped versions of much of the existing content in the explored areas. Therefore, the two former methods generally account for improvements which can be implemented upon these predecessors throughout development, which are primarily focussed on enhancing the user's experience. Myself and the discussed group of experienced others are a credible source for this, considering the scale of the project.\\

	Randomly assigning requirements to development would be irresponsible. To help better-address the importance of the components of the required functionality of a system, requirements should be analysed using some form of hierarchical tool which highlights a clearer path for the development process. In this case, the Must-Have/Should-Have/Could-Have/Won't-Have (MoSCoW) methodology is selected for this purpose. This helps differentiate between what functionality is essential to make the system run as intended, what functionality is required for optimization of the system, and what is required for additional enhancements. This approach generally shows functional requirements to average at the top-priority end and non-functional requirements to be distributed further down the hierarchy. With regards to the framework itself, note that:

	\begin{itemize}
	\setlength\itemsep{0cm}
		\item \textit{Must-Have} implementations refer to aspects of the system which must be present in order to make it basically functional and behave as it is intended and as the user desires;
		\item \textit{Should-Have} implementations refer to features of the system which should be implemented in order to make the essential features of the system more accessible and useable to the masses. They may also exist to improve efficiency, but [fly Under the Radar] and therefore go unnoticed by the user. They may offer additional functionality which makes the system more unique (or something of the sort) and therefore, more `creative' and attractive to users.
		\item \textit{Could-Have} implementations refer to aspects which may become present or relevant during the development of the former two. They may add additional functionality or usability to existing aspects or simply add final touches to the system overall. They are sometimes more contemporary.
		\item \textit{Won't Have} non-implementations refer to aspects of the system which aren't necessarily impossible or are of a nature which the system `can't have' but, which will probably be omitted or postponed due to constraints such as time, man-power, technical ability, finance, etc.
	\end{itemize}

	There is no formal client base for this system which means that no face-to-face client-oriented interviews or surveys can take place with regards to determining specific user desires. However, the aforementioned three-method protocol leads to the subsequently discussed requirements. To reiterate, the following list of requirements is generated through discussion of the desires of [1] an existing group of users of systems alike (some of whom are developers), [2] myself, another existing user of the sort, and [3] analysis of other services. This list is of course not exhaustive, as there is always room for perpetual development. However, it does account for every currently visible desire and possibility given the constraints.

	\subsection{Requirements}

	As discussed, it's important at this stage to clearly differentiate between essential functionality for the foundations of the system, generally accounted for in \textit{functional requirements}; and functionality which more contemporary, generally accounted for in \textit{non-functional} requirements. This is often considered `making things \textit{work} and making things \textit{relevant}'. It is also a useful method from which to infer associations between elements of the design and construction stages of development.Retaining these requirements, their position, and hierarchy at the center of development throughout the process allows accurate amendment of software and/or requirements along the way as more functionalities and possibilities become apparent and tangible or alternatively, further from reach.

		\subsubsection{Functional}

	Requirements here are presented in a numeric format. This does not refer to any hierarchical order, it simply creates a reference point for user stories. Hierarchy remains determined by the MoSCoW methodology. So forth, as required by the three discussed groups, users and the system \textit{must have} the ability to:

	\begin{enumerate}
	\setlength\itemsep{0cm}
		\item Access the user's current location upon various types of request
		\item View an `overview' of recommended routes
		\item Accept inputs related to user `abilities' and compute and display results based on them
		\item Accept inputs related to user `equipment' and compute and display results based on them
		\item View a weather forecast breakdown using real weather data
		\item Display an Ordnance Survey map
	\end{enumerate}

	Additionally, users and the system \textit{should have} the ability to:

	\begin{enumerate}
	\setlength\itemsep{0cm}
		\item On OS Map, alternate between `OS Leisure' (primary/default), `OS Road', and `OS Outdoor' topographical structures
		\item On OS Map, display grid markers at center-pan
	\end{enumerate}

	Additionally, given the various constraints, users and the system \textit{could have} the ability to:

	\begin{enumerate}
	\setlength\itemsep{0cm}
		\item Create an account and have their data stored
		\item Protect user data using an appropriate authentication and security system
		\item Store and re-use user attributes such as the aforementioned `ability' and `equipment cache'
		\item Choose `priority attributes' relevant to their routes which are stored and used to generate more relevant route recommendations etc.
	\end{enumerate}

	Additionally, users and the system \textit{won't have} the ability to:

	\begin{enumerate}
	\setlength\itemsep{0cm}
		\item Store route-relevant data and display them as `historically completed routes', or something of the sort, under the aforementioned `overview'
	\end{enumerate}

		\subsubsection{Non-Functional}

	To ensure a stable, trustworthy, useable and relevant system; users and the system will have the ability to / have the capacity to:

	\begin{enumerate}
	\setlength\itemsep{0cm}
		\item Be written in such a manner which allows full compatibility with the majority of web browsers (this could effect markup languages, font packages, JavaScript libraries, etc.)
		\item Be written in such a manner which allows full scalability between desktop and mobile use
		\item Offer the user various descriptive background [pieces], otherwise referred to as `key's and `suggested reading's which provide aid to users' background knowledge, understanding and decision making
		\item Offer the appropriate combination of relevant graphic design and actual functionality
		\item Adhere to the appropriate accessibility standards, with reference primarily to graphic design and system structure
	\end{enumerate}

		\subsubsection{User Stories}

	`User stories' are often an effective method of contextualizing requirements and presenting how they can and will be interpreted and used by the final user. Putting yourself in the hypothetical context of a user is a good method of delivering basic feedback on how to apply a solution to a requirement. Additionally, it helps identify possible negations or irrelevant content. So forth, the user stories found in Figure A4 in Appendix 1 follow the syntax: ``I wish to be able to $<$interact\_with\_feature$>$ in anticipation of $<$returned\_result$>$ which will provide me with $<$payoff$>$''. Development of the solution to these user stories listed in the figure are prioritized using the metric seen in the \textit{Priority} column.

	\subsection{User Accessibility}

	This system is available as a website application, it could be accessed and used by anyone with access to the internet. Therefore, it is to some degree essential that users of all types are welcomed to the site, even if they have no prior knowledge of the industry or field. More importantly, the graphic design and methods of interfacing in the user interface should be accommodating of users with possible imparities, such as a [severe mental imparity], by taking a user-centered approach which considers visible accessibility controls which allow these users to interface with the system more easily. Accessibility does not however only refer to disability. Ensuring optimal accessibility also accounts for factors such as implementing clean and concise CSS and graphic design for simple and intuitive navigation, stimulation, relevance, etc. These approaches are widely considered and implemented.\\

	In context, whether using the desktop or scaled mobile version of the site, the user interface implements many `friendly' universal standards. For example, it primarily uses the readable and elegant Sans-Serif font Audi AG and Serif font Garamond, with suitable font-sizing and color etc. Hue, saturation and luminance are appropriately considered/assigned to create [maximal optical pleasure] and proper readability with the correct contrast between elements. The well respected font-family Font Awesome, which is universally recognised, to display various standard symbols. In theory, these considerations lead to more efficient understanding and transitioning between elements of the site.

\newpage

\section{System Design}\label{ch5}

	\subsection{System Architecture}

	\subsection{User Interface}

		\subsubsection{Logical Design}

		\subsubsection{Graphic Design \& Communication}

		\subsubsection{Interface Tree}

		\subsubsection{Use Case Diagram}

		\subsubsection{Primary Use Case Examples}

	\subsection{Data Structure}

\newpage

\section{System Construction \& Implementation}\label{ch6}

	\subsection{Development \& Languages}

		\subsubsection{Development Environment}

		\subsubsection{Front-End \& User Interface}

		\subsubsection{Semi-Rear-End \& Supporting Data}

	\subsection{Application Programming Interfaces (APIs) \& Libraries}

		\subsubsection{Geolocation}

		\subsubsection{Ordnance Survey}

		OS Road (1 : 250 000), OS Landranger (1 : 50 000), OS Explorer (1 : 25 000)\\

		\subsubsection{Open Weather}

		\subsubsection{Leaflet}

		Leaflet JavaScript library for maps

		\subsubsection{Mapbox}

		Mapbox script for GPX-GeoJSON conversion

		\subsubsection{Chart.js}

		Chart.js JavaScript library for ranger

	\subsection{Project Management}

		\subsubsection{Life-Cycle \& Timing}

		\subsubsection{Supporting Tools}

\newpage

\section{Evaluation}\label{ch7}

	\subsection{Requirements}

	\subsection{Design, Construction \& Implementation}

		\subsubsection{Heuristic Evaluation}

		\subsubsection{User Acceptance (UAT), Accessibility \& Usability}

		\subsubsection{`Prototyping' \& Constant Evaluation}

	\subsection{Additional Testing}

		\subsubsection{`Test-Driven' Development}

		\subsubsection{Unit Testing}

	\subsection{Constrictions}

	Restricted to Glen Coe and Glen Etive

\newpage

\section{Development Conclusions}\label{ch8}

	\vspace{\fill}

	\begin{center}
		\textsc{A Michael Mann Production}\\
		\textsc{\textcopyright 1984 Orion Pictures Corporation}\\
		\textsc{\small{All Rights Reserved}}\\
		\textsc{Dolby Stereo}\texttrademark \textsc{In Selected Theatres}
	\end{center}

\newpage

\pagenumbering{Roman}

\fancyhead[L]{\textsc{APPENDICES}}

\section*{Appendices}

	Testing questionnaires, test cards (exp. result etc.), heuristic evaluation of system

	\subsection*{Appendix 1: Foundational Material}

		\subsubsection*{Figure A1: Data} 

	\begin{center}
		\tiny
		\renewcommand{\arraystretch}{1.5}
	\begin{longtable}{p{4cm}p{5cm}p{4cm}}
		\hline
		\hline
		\textsc{Data/Type} & \textsc{Purpose/Desc.} & \textsc{Location}\\
		\hline
		\hline
		\multicolumn{3}{c}{\textsc{Location (Geolocation API)}}\\
		\hline
		\fbox{latitude}\newline Number & Provide geographical data relative to the user's machine & Geolocation API\\
		\hline
		\fbox{longitude}\newline Number & Provide geographical data relative to the user's machine & Geolocation API\\
		\hline
		\multicolumn{3}{c}{\textsc{Regional (Research$^1$)}}\\
		\hline
		\fbox{county}\newline List (of Dict.) & Store data on each county in Scotland & JSON [1]\\
		\fbox{name}\newline $\in$ county\newline String & Provide name of each county in Scotland & JSON [1]\\
		\fbox{lat}\newline $\in$ county\newline Number & Provide latitude of each county in Scotland & JSON [1]\\
		\fbox{lon}\newline $\in$ county\newline Number & Provide longitude of each county in Scotland & JSON [1]\\
		\fbox{region}\newline List (of Dict.) & Store data on each region in Scotland & JSON [1]\\
		\fbox{name}\newline $\in$ region\newline String & Provide name of each region in Scotland & JSON [1]\\
		\fbox{subregion}\newline $\in$ region\newline List (of Dict.) & Store data on each sub-region in Scotland & JSON [1]\\
		\fbox{name}\newline $\in$ subregion $\subset$ region\newline String & Provide name of each subregion in Scotland & JSON [1]\\
		\fbox{subsubregion}\newline $\in$ subregion $\subset$ region\newline List (of Dict.) & Store data on each sub-sub-region in Scotland & JSON [1]\\
		\fbox{name}\newline $\in$ subsubregion $\subset$ subregion $\subset$ region\newline String & Provide name of each sub-sub-region in Scotland & JSON [1]\\
		\fbox{lat}\newline $\in$ subsubregion $\subset$ subregion $\subset$ region\newline Number & Provide latitude of each sub-sub-region in Scotland & JSON [1]\\
		\fbox{lon}\newline $\in$ subsubregion $\subset$ subregion $\subset$ region\newline Number & Provide longitude of each sub-sub-region in Scotland & JSON [1]\\
		\hline
		\multicolumn{3}{c}{\textsc{Landmass (Research)}}\\
		\hline
		\fbox{landmass}\newline List (of Dict.) & Store data on each landmass in Scotland, within the sample size & JSON [1]\\
		\fbox{name}\newline $\in$ landmass\newline String & Provide name of each landmass in Scotland, within the sample size & JSON [1]\\
		\fbox{type}\newline $\in$ landmass\newline String & Provide type of each landmass in Scotland, within the sample size (see Home Page $\rightarrow$ Overview for possible values) & JSON [1]\\
		\fbox{subtype}\newline $\in$ landmass\newline String & Provide sub-type of each landmass in Scotland, within the sample size (see Home Page $\rightarrow$ Overview for possible values) & JSON [1]\\
		\fbox{subsubtype}\newline $\in$ landmass\newline String & Provide sub-sub-type of each landmass in Scotland, within the sample size & JSON [1]\\
		\fbox{munro},\newline \fbox{munrotop},\newline \fbox{corbett},\newline \fbox{corbetttop}$^2$\newline $\in$ landmass\newline List (of Dict.) & Store data on each Munro, Munro Top, Corbett and Corbett Top, respectively, present on each landmass in Scotland, within the sample size & JSON [1]\\
		\fbox{name}\newline $\in$ \{munro $\lor$ munrotop $\lor$ corbett $\lor$ corbetttop\} $\subset$ landmass\newline String & Provide name of each Munro, Munro Top, Corbett and Corbett Top present on each landmass in Scotland, within the sample size & JSON [1]\\
		\fbox{lat}\newline $\in$ \{munro $\lor$ munrotop $\lor$ corbett $\lor$ corbetttop\} $\subset$ landmass\newline Number & Provide latitude of each Munro, Munro Top, Corbett and Corbett Top present on each landmass in Scotland, within the sample size & JSON [1]\\
		\fbox{lon}\newline $\in$ \{munro $\lor$ munrotop $\lor$ corbett $\lor$ corbetttop\} $\subset$ landmass\newline Number & Provide longitude of each Munro, Munro Top, Corbett and Corbett Top present on each landmass in Scotland, within the sample size & JSON [1]\\
		\fbox{OSgrid}\newline $\in$ \{munro $\lor$ munrotop $\lor$ corbett $\lor$ corbetttop\} $\subset$ landmass\newline String & Provide OS Grid Coordinates of each Munro, Munro Top, Corbett and Corbett Top present on each landmass in Scotland, within the sample size & JSON [1]\\
		\fbox{elevation}\newline $\in$ \{munro $\lor$ munrotop $\lor$ corbett $\lor$ corbetttop\} $\subset$ landmass\newline Number & Provide elevation (ft) of each Munro, Munro Top, Corbett and Corbett Top present on each landmass in Scotland, within the sample size & JSON [1]\\
		\fbox{prominence}\newline $\in$ \{munro $\lor$ munrotop $\lor$ corbett $\lor$ corbetttop\} $\subset$ landmass\newline Number & Provide prominence (ft) of each Munro, Munro Top, Corbett and Corbett Top present on each landmass in Scotland, within the sample size & JSON [1]\\
		\fbox{isolation}\newline $\in$ \{munro $\lor$ munrotop $\lor$ corbett $\lor$ corbetttop\} $\subset$ landmass\newline Number & Provide isolation (ft) of each Munro, Munro Top, Corbett and Corbett Top present on each landmass in Scotland, within the sample size & JSON [1]\\
		\fbox{summit}\newline $\in$ \{munro $\lor$ munrotop $\lor$ corbett $\lor$ corbetttop\} $\subset$ landmass\newline String & Provide name/type of summit feature of each Munro, Munro Top, Corbett and Corbett Top present on each landmass in Scotland, within the sample size & JSON [1]\\
		\fbox{image}\newline $\in$ \{munro $\lor$ munrotop $\lor$ corbett $\lor$ corbetttop\} $\subset$ landmass\newline String & Provide suffix of the image file path of each Munro, Munro Top, Corbett and Corbett Top present on each landmass in Scotland, within the sample size & JSON [1]\\
		\fbox{corrie}$^3$\newline $\in$ landmass\newline List (of Dict.) & Store data on each Corrie present on each landmass in Scotland, within the sample size & JSON [1]\\
		\fbox{name}\newline $\in$ corrie $\subset$ landmass\newline String & Provide name of each Corrie present on each landmass in Scotland, within the sample size & JSON [1]\\
		\fbox{gully}\newline $\in$ landmass\newline Boolean  & Provide true/false value based on whether each landmass in Scotland has one or more gullies, within the sample size & JSON [1]\\
		\fbox{lochain}\newline $\in$ landmass\newline Boolean & Provide true/false value based on whether each landmass in Scotland has one or more lochains, within the sample size & JSON [1]\\
		\fbox{waterfall}\newline $\in$ landmass\newline Boolean & Provide true/false value based on whether each landmass in Scotland has one or more waterfalls, within the sample size & JSON [1]\\
		\fbox{peatbog}\newline $\in$ landmass\newline Boolean & Provide true/false value based on whether each landmass in Scotland has one or more peat bogs or peat hag fields, within the sample size & JSON [1]\\
		\fbox{mudbog}\newline $\in$ landmass\newline Boolean & Provide true/false value based on whether each landmass in Scotland has one or more (usually unnatural) mud bogs, within the sample size & JSON [1]\\
		\fbox{boulderfield}\newline $\in$ landmass\newline Boolean & Provide true/false value based on whether each landmass in Scotland has one or more boulder fields, within the sample size & JSON [1]\\
		\fbox{scree}\newline $\in$ landmass\newline Boolean & Provide true/false value based on whether each landmass in Scotland has one or more scree slopes or deposits, within the sample size & JSON [1]\\
		\fbox{shoulder}\newline $\in$ landmass\newline Boolean & Provide true/false value based on whether each landmass in Scotland has one or more shoulders, within the sample size & JSON [1]\\
		\fbox{arete}\newline $\in$ landmass\newline Boolean & Provide true/false value based on whether each landmass in Scotland has one or more ar\^{e}tes, within the sample size & JSON [1]\\
		\fbox{humanfeatures}\newline $\in$ landmass\newline List (of Dict.) & Store data on each human feature present on each landmass in Scotland, within the sample size & JSON [1]\\
		\fbox{name}\newline $\in$ humanfeatures $\subset$ landmass\newline String & Provide name of each human feature present on each landmass in Scotland, within the sample size & JSON [1]\\
		\fbox{type}\newline $\in$ humanfeatures $\subset$ landmass\newline String & Provide type of each human feature present on each landmass in Scotland, within the sample size & JSON [1]\\
		\fbox{parentlandmass}\newline $\in$ landmass\newline String & Provide name of the parent landmass of each landmass in Scotland, within the sample size & JSON [1]\\
		\fbox{parentpeak}\newline $\in$ landmass\newline String & Provide name of the parent peak of each landmass in Scotland, within the sample size & JSON [1]\\
		\fbox{region}\newline $\in$ landmass\newline String & Provide name of the region in which each landmass in Scotland is located, within the sample size & JSON [1]\\
		\fbox{subregion}\newline $\in$ landmass\newline String & Provide name of the sub-region in which each landmass in Scotland is located, within the sample size & JSON [1]\\
		\fbox{informalregion}\newline $\in$ landmass\newline String & Provide name of the informal region in which each landmass in Scotland is located, within the sample size & JSON [1]\\
		\hline
		\multicolumn{3}{c}{\textsc{Route}}\\
		\hline
		\fbox{route}\newline $\in$ landmass\newline List (of Dict.) & Store data on each present on each landmass in Scotland, within the sample size and the particular routes selected & JSON [1]\\
		\fbox{name}\newline $\in$ route $\subset$ landmass\newline String & Provide name of each route present on each landmass in Scotland, within the sample size and the particular routes selected & JSON [1]\\
		\fbox{distance}\newline $\in$ route $\subset$ landmass\newline Number & Provide distance (mi) of each route present on each landmass in Scotland, within the sample size and the particular routes selected & JSON [1]\\
		\fbox{elevationgain}\newline $\in$ route $\subset$ landmass\newline Number & Provide total elevation gain (ft) of each route present on each landmass in Scotland, within the sample size and the particular routes selected & JSON [1]\\
		\fbox{stdtime}\newline $\in$ route $\subset$ landmass\newline Number & Provide standard time (hrs) taken by a normie to complete each route present on each landmass in Scotland, within the sample size and the particular routes selected & JSON [1]\\
		\fbox{type}\newline $\in$ route $\subset$ landmass\newline List (of Strings) & Provide array of descriptions of the types/purposes of each route present on each landmass in Scotland, within the sample size and the particular routes selected & JSON [1]\\
		\fbox{stage}\newline $\in$ route $\subset$ landmass\newline List (of Strings) & Provide array of descriptions of the stages of each route present on each landmass in Scotland, within the sample size and the particular routes selected & JSON [1]\\
		\fbox{terraintype}\newline $\in$ route $\subset$ landmass\newline List (of Strings) & Provide array of descriptions of the terrain types of each route present on each landmass in Scotland, within the sample size and the particular routes selected & JSON [1]\\
		\fbox{terraindiff}\newline $\in$ route $\subset$ landmass\newline List (of Strings) & Provide array of descriptions of the terrain difficulties of each route present on each landmass in Scotland, within the sample size and the particular routes selected & JSON [1]\\
		\fbox{gear}\newline $\in$ route $\subset$ landmass\newline List (of Strings) & Provide array of descriptions of the recommended gear required for each route present on each landmass in Scotland, within the sample size and the particular routes selected & JSON [1]\\
		\fbox{munro}\newline $\in$ route $\subset$ landmass\newline List (of Strings) & Provide array of the names of Munros on each route present on each landmass in Scotland, within the sample size and the particular routes selected & JSON [1]\\
		\fbox{munrotop}\newline $\in$ route $\subset$ landmass\newline List (of Strings) & Provide array of the names of Munro Tops on each route present on each landmass in Scotland, within the sample size and the particular routes selected & JSON [1]\\
		\fbox{corbett}\newline $\in$ route $\subset$ landmass\newline List (of Strings) & Provide array of the names of Corbetts on each route present on each landmass in Scotland, within the sample size and the particular routes selected & JSON [1]\\
		\fbox{corbetttop}\newline $\in$ route $\subset$ landmass\newline List (of Strings) & Provide array of the names of Corbett Tops on each route present on each landmass in Scotland, within the sample size and the particular routes selected & JSON [1]\\
		\fbox{GPX}\newline $\in$ route $\subset$ landmass\newline String & Provide suffix of the GPX file path of each route present on each landmass in Scotland, within the sample size and the particular routes selected & JSON [1]\\
		\hline
		\multicolumn{3}{c}{\textsc{Ability}}\\
		\hline
		\fbox{elementshill}\newline List (of Dict.) & Store data on each attribute of a hill (see Home Page $\rightarrow$ Overview for possible values) & JSON [2]\\
		\fbox{elementsroute}\newline List (of Dict.) & Store data on each attribute of a route (see Home Page $\rightarrow$ Overview for possible values) & JSON [2]\\
		\fbox{summitfeats}\newline List (of Dict.) & Store data on each type of summit feature (see Home Page $\rightarrow$ Overview for possible values) & JSON [2]\\
		\fbox{type}\newline List (of Dict.) & Store data on each type of traverse/sport you can do/practice on a hill its routes (see Home Page $\rightarrow$ Overview for possible values) & JSON [2]\\
		\fbox{stage}\newline List (of Dict.) & Store data on each of the traverse stages included on a hill and its routes (see Home Page $\rightarrow$ Overview for possible values) & JSON [2]\\
		\fbox{terraintype}\newline List (of Dict.) & Store data on each type of terrain on a hill and its routes (see Home Page $\rightarrow$ Overview for possible values) & JSON [2]\\
		\fbox{terraindiff}\newline List (of Dict.) & Store data on each relative terrain difficulty factor of a hill and its routes (see Home Page $\rightarrow$ Overview for possible values) & JSON [2]\\
		\fbox{name}$^4$\newline $\in$ \{elementshill $\lor$ elementsroute $\lor$ summitfeats $\lor$ type $\lor$ stage $\lor$ terraintype $\lor$ terraindiff\}\newline String & Provide name of each attribute of a hill, attribute of a route, type of summit feature, type of traverse/sport you can do/practice, type of traverse stages, type of terrain, relative terrain difficulty factor & JSON [2]\\
		\fbox{desc}$^5$\newline $\in$ \{elementshill $\lor$ elementsroute $\lor$ type $\lor$ stage $\lor$ terraintype\}\newline String & Provide description of each attribute of a hill, attribute of a route, type of traverse/sport you can do/practice, type of traverse stages, type of terrain & JSON [2]\\
		\fbox{image}$^6$\newline $\in$ \{summitfeats $\lor$ type $\lor$ stage $\lor$ terraintype $\lor$ terraindiff\}\newline String & Provide suffix of the image file path of each type of summit feature, type of traverse/sport you can do/practice, type of traverse stages, type of terrain, relative terrain difficulty factor & JSON [2]\\
		\hline
		\multicolumn{3}{c}{\textsc{Equipment}}\\
		\hline
		\fbox{packs}\newline List (of Dict.) & Store data on each type of pack, travel equipment and their accessories, relevant to outdoor activity (see Home Page $\rightarrow$ Overview for possible values) & JSON [2]\\
		\fbox{technical}\newline List (of Dict.) & Store data on each type of technical equipment and their accessories, relevant to outdoor activity (see Home Page $\rightarrow$ Overview for possible values) & JSON [2]\\
		\fbox{shoes}\newline List (of Dict.) & Store data on each type of footwear and their accessories, relevant to outdoor activity (see Home Page $\rightarrow$ Overview for possible values) & JSON [2]\\
		\fbox{clothing}\newline List (of Dict.) & Store data on each type of clothing and their accessories, relevant to outdoor activity (see Home Page $\rightarrow$ Overview for possible values) & JSON [2]\\
		\fbox{name}$^7$\newline $\in$ \{packs $\lor$ technical $\lor$ shoes $\lor$ clothing\}\newline String & Provide name of each pack, technical component, shoe, item of clothing & JSON [2]\\
		\fbox{desc}$^8$\newline $\in$ \{packs $\lor$ technical $\lor$ shoes\}\newline String & Provide description of each pack, technical component, shoe & JSON [2]\\
		\fbox{comp}$^9$\newline $\in$ \{shoes\}\newline List & Provide array of compatibility options for each shoe (boots and crampons, etc.) & JSON [2]\\
		\fbox{feat}$^9$\newline $\in$ \{shoes\}\newline List & Provide array of features available with each shoe & JSON [2]\\
		\fbox{adv}$^9$\newline $\in$ \{shoes\}\newline List & Provide array of advantages of each shoe, item of clothing & JSON [2]\\
		\fbox{dangers}$^9$\newline $\in$ \{shoes\}\newline List & Provide array of dangers of each shoe & JSON [2]\\
		\fbox{image}$^{10}$\newline $\in$ \{packs $\lor$ technical $\lor$ shoes $\lor$ clothing\}\newline String & Provide suffix of the image file path of each pack, technical component, shoe, item of clothing & JSON [2]\\
		\hline
		\multicolumn{3}{c}{\textsc{Weather (OpenWeather API)}}\\
		\hline
		\fbox{dt} -- Day\newline $\in$ daily\newline Number & Provide date and time of weather occurrence for the day & Computed from OpenWeather \href{https://openweathermap.org/api/one-call-api}{One Call API 1.0}, as JavaScript date conversion of:\newline $1000(\mathtt{day.dt})$\\
		\fbox{icon} -- Day\newline $\in$ \{weather $\subset$ daily\}\newline String & Provide a graphical icon representing the average weather condition for the day & Computed from OpenWeather One Call API 1.0, associating existing symbols with custom ones\\
		\fbox{main} -- Day\newline $\in$ \{weather $\subset$ daily\}\newline String & Provide a word or phrase which summarises the average weather condition for the day & OpenWeather One Call API 1.0\\
		\fbox{description} -- Day\newline $\in$ \{weather $\subset$ daily\}\newline String & Provide a phrase or short sentence which summarises the average weather condition for the day & OpenWeather One Call API 1.0\\
		\fbox{max} -- Day\newline $\in$ \{temp $\subset$ daily\}\newline Number & Provide the value of the maximum forecasted temperature for the day & OpenWeather One Call API 1.0\\
		\fbox{min} -- Day\newline $\in$ \{temp $\subset$ daily\}\newline Number & Provide the value of the minimum forecasted temperature for the day & OpenWeather One Call API 1.0\\
		\fbox{day} -- Day\newline $\in$ \{feels\_like $\subset$ daily\}\newline Number & Provide the value of the daytime `feels like' temperature for the day & OpenWeather One Call API 1.0\\
		\fbox{night} -- Day\newline $\in$ \{feels\_like $\subset$ daily\}\newline Number & Provide the value of the nighttime `feels like' temperature for the day & OpenWeather One Call API 1.0\\
		\fbox{pop} -- Day\newline $\in$ daily\newline Number & Provide a probability value between 0 and 1 which represents the chance of precipitation for the day & Computed from OpenWeather One Call API 1.0, as:\newline $100(\mathtt{day.pop})$\\
		\fbox{wind\_deg} -- Day\newline $\in$ daily\newline Number & Provide a value representing the average angle from which the wind is `blowing' for the day & OpenWeather One Call API 1.0\\
		\fbox{wind\_deg} (for Description) -- Day\newline $\in$ daily\newline Number & Provide a value representing the average angle from which the wind is `blowing' for the day, which can be compared to a defined group of angular ranges to determine a description & Computed from OpenWeather One Call API 1.0, based on angular ranges\\
		\fbox{wind\_deg} (for Symbol) -- Day\newline $\in$ daily\newline Number & Provide a value representing the average angle from which the wind is `blowing' for the day, which can be compared to a defined group of angular ranges to determine a suitable symbol & Computed from OpenWeather One Call API 1.0, as:\newline $(-45)+180+\mathtt{day.wind\_deg}$\\
		\fbox{wind\_speed} -- Day\newline $\in$ daily\newline Number & Provide the value of the average wind speed for the day & OpenWeather One Call API 1.0\\
		\fbox{pressure} -- Day\newline $\in$ daily\newline Number & Provide the value of the average air pressure for the day & OpenWeather One Call API 1.0\\
		\fbox{humidity} -- Day\newline $\in$ daily\newline Number & Provide the value of the average humidity for the day & OpenWeather One Call API 1.0\\
		\fbox{dew\_point} -- Day\newline $\in$ daily\newline Number & Provide the value of the average dew point for the day & OpenWeather One Call API 1.0\\
		\fbox{uvi} -- Day\newline $\in$ daily\newline Number & Provide the value from an index representing the average ultraviolet radiation emitted for the day & OpenWeather One Call API 1.0\\
		\fbox{sunrise} -- Day\newline $\in$ daily\newline Number & Provide time of sunrise for the day & Computed from OpenWeather One Call API 1.0, as JavaScript date conversion of:\newline $1000(\mathtt{day.sunrise})$\\
		\fbox{sunset} -- Day\newline $\in$ daily\newline Number & Provide time of sunset for the day & Computed from OpenWeather One Call API 1.0, as JavaScript date conversion of:\newline $1000(\mathtt{day.sunset})$\\



		\fbox{dt} -- Hour\newline $\in$ hourly\newline Number & Provide date and time of weather occurrence for the hour & Computed from OpenWeather \href{https://openweathermap.org/api/one-call-api}{One Call API 1.0}, as JavaScript date conversion of:\newline $1000(\mathtt{hour.dt})$\\
		\fbox{icon} -- Hour\newline $\in$ \{weather $\subset$ hourly\}\newline String & Provide a graphical icon representing the average weather condition for the hour & Computed from OpenWeather One Call API 1.0, associating existing symbols with custom ones\\
		\fbox{temp} -- Hour\newline $\in$ hourly\newline Number & Provide the value of the average forecasted temperature for the hour & OpenWeather One Call API 1.0\\
		\fbox{feels\_like} -- Hour\newline $\in$ hourly\newline Number & Provide the value of the average `feels like' temperature for the hour & OpenWeather One Call API 1.0\\
		\fbox{pop} -- Hour\newline $\in$ hourly\newline Number & Provide a probability value between 0 and 1 which represents the chance of precipitation for the hour & Computed from OpenWeather One Call API 1.0, as:\newline $100(\mathtt{hour.pop})$\\
		\fbox{wind\_deg} -- Hour\newline $\in$ hourly\newline Number & Provide a value representing the average angle from which the wind is `blowing' for the hour & OpenWeather One Call API 1.0\\
		\fbox{wind\_deg} (for Description) -- Hour\newline $\in$ hourly\newline Number & Provide a value representing the average angle from which the wind is `blowing' for the hour, which can be compared to a defined group of angular ranges to determine a description & Computed from OpenWeather One Call API 1.0, based on angular ranges\\
		\fbox{wind\_deg} (for Symbol) -- Hour\newline $\in$ hourly\newline Number & Provide a value representing the average angle from which the wind is `blowing' for the hour, which can be compared to a defined group of angular ranges to determine a suitable symbol & Computed from OpenWeather One Call API 1.0, as:\newline $(-45)+180+\mathtt{hour.wind\_deg}$\\
		\fbox{wind\_speed} -- Hour\newline $\in$ hourly\newline Number & Provide the value of the average wind speed for the hour & OpenWeather One Call API 1.0\\
		\fbox{wind\_gust} -- Hour\newline $\in$ hourly\newline Number & Provide the value of the average wind `gust' speed for the hour & OpenWeather One Call API 1.0\\
		\fbox{pressure} -- Hour\newline $\in$ daily\newline Number & Provide the value of the average air pressure for the hour & OpenWeather One Call API 1.0\\
		\fbox{humidity} -- Hour\newline $\in$ daily\newline Number & Provide the value of the average humidity for the hour & OpenWeather One Call API 1.0\\
		\fbox{dew\_point} -- Hour\newline $\in$ daily\newline Number & Provide the value of the average dew point for the hour & OpenWeather One Call API 1.0\\
		\fbox{visibility} -- Hour\newline $\in$ hourly\newline Number & Provide the value from an index representing the average visibility for the hour & OpenWeather One Call API 1.0\\
		\fbox{uvi} -- Hour\newline $\in$ hourly\newline Number & Provide the value from an index representing the average ultraviolet radiation emitted for the hour & OpenWeather One Call API 1.0\\
		\hline
		\multicolumn{3}{p{13.75cm}}{$^1$: Any component labelled `research' refers to my personal iterations through Ordnance Survey maps and in-person field trips to collect geographical information.\newline $^2$: Munro, Munro Top, Corbett, Corbett Top variables are each their own lists containing different values however, the variables for each are identical and therefore are combined to save space. The only difference is Munros and Corbetts are included in weather location search, whereas Munro Tops and Corbett Tops are not as, well, who wants to know the weather on a Top?!\newline $^3$: As this list of dictionaries just contains single-type attributes (the names of the corries, it could simply be a list of strings however, it remains this way in anticipation of the subsequent addition of Corrie latitude and longitude data.)\newline $^4$: Each elementshill, elementsroute, summitfeats, type, stage, terraintype, terraindiff variables have their own name variable however, are included comprehensively here.\newline $^5$: Same situation as footnote 4 however, summitfeats and terraindiff do not include description variables\newline $^6$: Same situation as footnotes 4 and 5 however, elementshill and elementsroute do not include image variables\newline $^7$: Each packs, technical, shoes, clothing variables have their own name variable however, are included comprehensively here.\newline $^8$: Same situation as footnote 7 however, clothing does not include description variables\newline $^9$: Same situation as footnotes 7 and 8 however, packs, technical, clothing do not include compatibility, feature, advantages, dangers variables\newline $^{10}$: Same situation as footnotes 7, 8 and 9}\\
		\hline
	\end{longtable}
	\end{center}

		\subsubsection*{Figure A2: Heuristic Evaluation -- Walkhighalnds}

	\begin{center}
                \scriptsize
        	\begin{longtable}{p{7.5cm}p{0.5cm}p{0.5cm}p{4cm}}
                \textsc{Description} & \textsc{Vio.} & \textsc{Sev.} & \textsc{Proposed Solution}\\
                \hline
		\hline
		\multicolumn{4}{c}{\textsc{Basic Features}}\\
		\hline
			\textbf{Sparse Navigation}:\newline There is not much order to the site navigation and in places thee is a lack pf relevance and hierarchy to the options. For example, some sub-sets of pages/features are listed in navigation on the same `tier'; whether that being represented by the link having the same alignment as it's parent, or something of the sort & $\mathrm{H_{3}}$, $\mathrm{H_{4}}$ & $\mathrm{S_{3}}$ & Correctly use HTML and CSS to display the hierarchy of pages and features in this system\\
			\textbf{Difficult Navigation}:\newline Some essential literature relating to components such as ability and equipment, as featured in this system, is difficult to find and requires a long path of navigation through pages and their children. Often some literature is simply linked through an in-body hyperlink (i.e. in a paragraph) and not even allocated it's own title etc. & $\mathrm{H_{3}}$, $\mathrm{H_{4}}$ & $\mathrm{S_{3}}$ & Correctly use HTML and CSS to display the hierarchy of pages and features in this system. Ensure no relevance is lost in masses of text\\
		\hline
		\multicolumn{4}{c}{\textsc{Map Features}}\\
		\hline
			\textbf{Feature Separation}: The OS Maps displaying Munros and Corbetts etc. are displayed on different pages of the site, with navigation links between them. This means that interaction with features is limited to the point where all a user can do is view the information provided about it upon a click and follow further links from there. They cannot for example, toggle on/off different features while interacting with the map meaning isolating different features on a GPS route a user is viewing is impossible. & $\mathrm{H_{3}}$, $\mathrm{H_{7}}$ & $\mathrm{S_{3}}$ & Implement a toggle feature for access to different map feature when the user desires, as listed in requirements\\
		\hline
		\multicolumn{4}{c}{\textsc{Statistical Features}}\\
		\hline
		...\\
		\hline
		\hline
                \multicolumn{4}{l}{\textsc{Total Violations}: 3}\\
                \multicolumn{4}{l}{\textsc{Evaluator}: Lewis Britton}\\
                \multicolumn{4}{l}{\textsc{Platform(s)}: Brave Browser (Desktop), Brave Browser (Mobile)}\\
                \hline
                \caption{Heuristic Analysis}
        \end{longtable}
        \end{center}

\newpage

		\subsubsection*{Figure A3: Heuristic Evaluation -- Strava}

	\begin{center}
                \scriptsize
        	\begin{longtable}{p{7.5cm}p{0.5cm}p{0.5cm}p{4cm}}
                \textsc{Description} & \textsc{Vio.} & \textsc{Sev.} & \textsc{Proposed Solution}\\
                \hline
		\hline
		\multicolumn{4}{c}{\textsc{Basic Features}}\\
		\hline
			\textbf{Weather Misuse}:\newline Weather data is only available as part of complete routes. I.e. activities which are complete and have been saved. Strava use a weather API so it is hard to understand why they would bot integrate this as a basic summary feature on the home page or on a profile etc. Historic weather data, i.e. that printed on complete routes, is the most irrelevant use of such as, come on, the user already knows what the weather was like on their routes and if they want to show off a 20hr hike in a snow storm or something, upload a picture & $\mathrm{H_{3}}$ & $\mathrm{S_{3}}$ & Include a comprehensive weather system, as listed in requirements\\
		\hline
		\multicolumn{4}{c}{\textsc{Map Features}}\\
		\hline
			\textbf{Map Layers}:\newline Although the Mapbox map seen throughout Strava is used very effectively in hosting many dynamic layers such as customizable GPX routes with terrain' and athlete-specific attributes, it is just a basic map with contours, not a full topographical map. This significantly limits interaction with map features and interpretation as it simply doesn't show many of the relevant aspects to hiking & $\mathrm{H_{2}}$, $\mathrm{H_{3}}$ & $\mathrm{S_{3}}$ & Utilize the OS Map API instead of a more basic one to ensure full access to topographical mapping, as listed in requirements\\
		\hline
		\multicolumn{4}{c}{\textsc{Statistical Features}}\\
		\hline
		...\\
		\hline
		\hline
                \multicolumn{4}{l}{\textsc{Total Violations}: 2}\\
                \multicolumn{4}{l}{\textsc{Evaluator}: Lewis Britton}\\
                \multicolumn{4}{l}{\textsc{Platform(s)}: Brave Browser (Desktop), Brave Brows
er (Mobile)}\\
                \hline
                \caption{Heuristic Analysis}
        \end{longtable}
        \end{center}

\newpage

		\subsubsection*{Figure A4: User Stories, Prioritization \& Solutions}
			
		\begin{center}
			\scriptsize
		\begin{longtable}{p{4cm}lp{1.5cm}lp{4cm}}
			\textsc{Scenario} & \textsc{Req.} & \textsc{Relevance} & \textsc{Priority*} & \textsc{Solution}\\
			\hline
			\hline
			\multicolumn{5}{c}{\textsc{Functional -- Must-Have}}\\
			\hline
			``I wish to be able to [use the current location of my device] in anticipation of [seeing my location printed on a map] which would allow me to [navigate towards map features relative to my location]'' & 1 & UI\newline OS Maps & 1 & Utilize \textit{Geolocation} API to fetch location of user's device\\
			``I wish to be able to [use the current location of my device] in anticipation of [weather forecast results for my location] which would allow me to [quick-access my weather forecast, likely while on-route]'' & 1, 5 & UI\newline Weather Service & 1 & Utilize \textit{Geolocation} API to fetch location of user's device and place in weather function\\
			``I wish to be able to [quickly view a summary of suggested routes] which would allow me to [see all the relevant attributes of a route on one return] and therefore, [select routes with less thought, time and effort]'' & 2 & UI\newline Route Summary & 1 & Include a section on the home page with this/these information/statistics\\
			``I wish to be able to [include attributes related to my ability] which contribute to my [suggested routes] therefore, making them [more relevant to my needs]'' & 3 & UI\newline Ability Service & 1 & Include a section which accepts relevant user inputs and returns route suggestions as required\\
			``I wish to be able to [include attributes related to the equipment I own] which contribute to my [suggested routes] therefore, making them [more relevant to my needs]'' & 4 & UI\newline Equipment Service & 1 & Include a section which accepts relevant user inputs and returns route suggestions as required\\
			``I wish to be able to [seek locations by their name] in anticipation of the return of [weather forecast results related to their coordinates] which would allow me to [plan my routes more effectively]'' & 5 & UI\newline Weather Service & 1 & Include a section which accepts location-based inputs, whether this be pre-defined coordinates of locations or the user's current location as discussed previously, to display weather forecast results for these locations\\
			``I wish to be able to [view an Ordance Survey map with OS Road (1 : 250 000), OS Landranger (1 : 50 000), and OS Explorer (1 : 25 000) layers simultaneously, alternative upon zoom] in anticipation of [viewing initially basic features such as my location as a symbol on the map and center-pan coordinates adjacent to the map] therefore, [making location seeking more interactive and understanable]'' & 1, 6 & UI\newline OS Maps & 1 & Utilize \textit{Ordnance Survey} API to host the map\newline Utilize \textit{Leaflet} JavaScript library to initialize and interact with the map\\
			``I wish to have the option to [interact with the OS map] which allows me to [toggle on/off display of features], allowing me to [view only specific features I wish to include on my routes]'' & 6 & UI\newline OS Map & 1 & Utilize \textit{Leaflet} JavaScript library to display and toggle on/off symbols representing the location of various geographical features such as Munros, Munro Tops, Corbetts, and Corbett Tops\\
			``I wish to have the option to [interact with the OS map] which allows me to [toggle on/off GPS routes based on search criteria] therefore, [making route results relevant to my subjective choice], as opposed to other factors such as recommendations based on my ability and equipment, or raw search'' & 6 & UI\newline OS Map & 1 & Utilize \textit{Leaflet} JavaScript library to display and toggle on/off GPS route (GPX data) layers over the map, relative to various search criteria\newline Utilize \textit{Mapbox GL-JS} JavaScript library to convert GeoJSON files to GPX\\
			``I wish to be able to [search for routes close to my current location] which returns the [closest relevant GPS overlay] to my location, making it easy for me to [make a fast decision if I just want a close project]'' & 6, 1 & UI\newline OS Map & 1 & Utilize \textit{Geolocation} API to fetch location of user's device and compare to the location of all listed projects\\
			\hline
			\multicolumn{5}{c}{\textsc{Functional -- Should-Have}}\\
			\hline
			``I wish to be able to [interact with the OS Map] in order to [change its layers] which will allow me to [interpret the features and contours of the land in different ways]'' & 1 & UI\newline OS Map & 2 & Include some form of input which \textit{Leaflet} JavaScript library to allow the user to change primary map layer\\
			``I wish to [view grid markers at the center of the OS Map] which, when panning, will [pinpoint the center coordinates displayed on the map] therefore, [making it easy to view, locate and pan to and from features and locations etc.]'' & 2 & UI\newline Map OS & 2 & Include basic HTML and CSS which displays reticle\\
			\hline
			\multicolumn{5}{c}{\textsc{Functional -- Could-Have}}\\
			\hline
			``I wish to [have my data stored] in the system so I can [access it down the line] to [use it in various site features]'' & 1 & Back-End & 3 & Implement a back-end database system to store and interact with user data\\
			``With regards to [data access], I wish to be [assured of appropriate security measures] implemented by the system controllers which will give me [peace of mind] regarding data protection & 2 & Back-End & 3 & Implement appropriate authentication and data protection measures in the system''\\
			``I wish to be able to [use and reuse attributes related to my account] in order to [tailor results to my own abilities and equipment], for example; allowing me to [gain a more accurate understanding of what I'm able to do] with less effort'' & 3 & UI\newline Back-End & 3 & Allow already-implemented features such as user ability and equipment services to access and use stored user data, as opposed to using basic inputs\\
			``I wish to be able to [choose `priority attributes'] relevant to data stored upon my previous routes which returns [more relevant route recommendations] to historic ones, relevant to what I have and haven't done based on my preference; allowing me to [more dynamically choose routes]'' & 4 & UI\newline Back-end & 3 & Implement appropriate measures\\
			\hline
			\multicolumn{5}{c}{\textsc{Functional -- Won't Have}}\\
			\hline
			``I wish to use a GPS device such as a watch or cellular device to [record GPX data] which can be used to [generate route information] and [display useful route records] which I can interpret and benefit/learn from'' & 1 & Back-End & 4 & (Will not) implement functionality to accept GPX uploads and transform relevant components into human readable information\\
			\hline
			\multicolumn{5}{c}{\textsc{Non-Functional}}\\
			\hline
			``I wish to be able to [use the site on any browser] therefore meaning I can [view all content] regardless of what browser platform I use, meaning I'll be able to [consistently interact with the functionality]'' & 1 & UI\newline Browser Support & 5 & Ensure all languages, packages, libraries, etc., are using the most versatile syntax and have aggregate browser support\\
			``I wish to be able to [us ethe site on any device platform] therefore meaning I can [view all content] regardless of what device I use, meaning I'll be able to [consistently interact with the functionality]'' & 2 & UI & 5 & Implement the correct CSS protocol to allow the site do dynamically scale to various device sizes; using appropriate adjustment and hiding/showing of content\\
			``I wish to [view additional] literature based upon [information regarding topics I wish to explore]; [enhancing my knowledge of the components of the sport and my decision making]'' & 3 & UI\newline Research & 5 & Implement appropriate bodies of text relevant to areas upon which they may be if use to the user\\
			``I wish to [view elements of graphic design] which [show relevance to the areas in which they appear] therefore, stimulating, engaging and immersing me further in the sport]'' & 4 & UI & 5 & Use CSS to implement the correct elements and principals of graphic design, primarily color schemes and shape/geometry, which are relevant to the outdoor context\newline Use graphic design tools (such as Serif PagePlus and GIMP) to create and implement relevant features such as logos, images and renders, which are relevant to the outdoor context\\
			``I wish to be able to optimally interact with the site when it comes to navigation and viewing experience, regardless of any impairity or things of the sort'' & 5 & UI\newline Research & 5 & Implement appropriate design protocol to make this experience as easy as possible\\
			\hline
			\hline
			\multicolumn{5}{l}{*: `Must-Have' = 1; `Should-Have' = 2; `Could-Have' = 3; Won't Have = 4; Non-Functional = 5}\\
			\hline
		\end{longtable}
		\end{center}

		Various Sources (2022)

	\subsection*{Appendix 2: Foundational Material}

		\subsubsection*{Author's Note}

			\textit{The following content is manufactured by myself in aid of basic understanding of the background to contexts, data and methods present within this study. It is presented as teaching material, in a format I would output if in such position.}
		\subsubsection*{Figure B1: Linguistics of `{\LaTeX}'}

		{\LaTeX} (or LaTeX, even latex (Donald E. Knuth's more recent installment of {\TeX})) is usually pronounced /la\textlengthmark t$\varepsilon$k/ (`lah') or /le\textsc{i}t$\varepsilon$k/ (`lei'/`lay') in English (that is, not with the /ks/ pronunciation English speakers normally associate with X, but with a /k/). The characters T, E, X in the name come from capital Greek letters tau, epsilon, and chi, as the name of {\TeX} derives from the Greek: $\tau\varepsilon\chi\nu\eta$ (skill, art, technique, precision); for this reason, Donald E. Knuth promotes a pronunciation of /t$\varepsilon$k/ (tekh) (that is, with a voiceless velar fricative as in Modern Greek, similar to the last sound of the German word ``Bach", the Spanish ``j" sound, or as ``ch'' in a Scottish `loch'). 

		%F\"{u}hrer
		\subsubsection*{Figure B2: Don. Knuth's Computer Modern Unicode (CMU) Font Family}

		\begin{table}[h]
			\scriptsize
			\renewcommand{\arraystretch}{1.25}
		\begin{center}
		\begin{tabular}{p{4cm}p{4cm}p{4cm}}
			\hline
			\multicolumn{1}{c}{\textbf{Serif}} & \multicolumn{1}{c}{\textbf{Sans Serif}} & \multicolumn{1}{c}{\textbf{Monospaced}}\\
			\hline
			CMU Serif Roman & \textsf{CMU Sans Serif} & \texttt{CMU Concrete}\\
			\textbf{CMU Serif Bold} & \sffamily \textbf{CMU Sans Serif Bold} & \\ 
				\textit{CMU Serif Italic} & & \ttfamily \textit{CMU Concrete Italic}\\
			\textsl{CMU Serif Oblique} & \sffamily \textsl{CMU Sans Serif Oblique} & \ttfamily \textsl{CMU Concrete Oblique}\\
				\textsc{CMU Serif Small Caps} & & \ttfamily \textsc{CMU Concrete Small Caps}\\
			\hline
				\multicolumn{3}{p{13cm}}{\textit{In the presence of traditionalists, a suitable alternative to Donald E. Knuth's Computer Modern Unicode font family may be considered: Andale Mono.}}\\
			\hline
		\end{tabular}
		\end{center}
		\end{table}

		\subsubsection*{Figure B3: Why My Pre-Title’s Right and You’re Wrong}

		I have received numerous comments which anyone would regard na\"{i}ve and under-educated regarding my pre-title of this study: \textit{AG436 Dissertation Coursework Assignment}. The argument originates in the `Coursework Assignment’ portion. People argue that a dissertation `is not’/`does not have’ an assignment. Not only is this poor characteristic recognition, it is semantically wrong. \textit{AG436: Dissertation} is a class just like any other. However under this class, there are no lectures, no tutorials and therefore no exams as there is no [taught] content. Do not confuse this with the class `having no content’ though. AG436’s content is apparent through literature of the student’s choice. Therefore, it is possible for a `coursework assignment’ to be based on this. Hence, any further comments are null.

\newpage

	\renewcommand\refname{Bibliography}

	\fancyhead[L]{\leftmark}

	\begin{thebibliography}{9}

	\bibitem{}
		Nielsen, J. (1994).
		\textsl{Heuristic Evaluation.}
		Usability Inspection Methods, John Wiley \& Sons

	\bibitem{}
		Walkhighlands. (2022).
		\textsl{Walkhighlands: Scotland walks and accommodation}
		Available At:
		\texttt{https://www.walkhighlands.co.uk/}
		(Accessed: 22/03/2022).

	\bibitem{}
		MDN Web Docs. (2022).
		\textsl{Geolocation API - MDN Web Docs}
		Available At:
		\texttt{https://developer.mozilla.org/en-US/docs/Web/API/Geolocation\_API}
		(Accessed: 08/04/2022).

	\bibitem{}
		Peakbagger.com (2022).
		\textsl{Search - Peakbagger.com}
		Available At:
		\texttt{https://www.peakbagger.com/Search.aspx}
		(Accessed: 03/04/2022).

	\bibitem{}
		Harold Street. (2022).
		\textsl{Mountain GPS Waypoints \& Hill Bagging Lists}
		Available At:
		\texttt{https://www.haroldstreet.org.uk/waypoints/}
		(Accessed: 03/04/2022).

	\end{thebibliography}

\end{document}
